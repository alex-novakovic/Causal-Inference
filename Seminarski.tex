\documentclass[12pt, a4paper]{article}

\usepackage{authblk}
\usepackage[utf8]{inputenc}
\usepackage[T2A]{fontenc}
\usepackage[serbianc]{babel}
\usepackage{hyperref}

\renewcommand\Authsep{\par}
\renewcommand\Authands{\par}

\title{Увод у каузално (узрочно) закључивање}
\author{Александра Новаковић 368/22}
\author{Милица Ињац 338/18}
\author{Бојан Корда 121/19}
\affil{Математички факултет, Универзитет у Београду}
\date{\today}

\begin{document}
\maketitle
\newpage

\tableofcontents
\newpage

\section{Фундаментални проблем}
    \subsection{Увод}
    \subsubsection{корелација не повлачи узрочност}
    \subsubsection{зашто је важно}
    \subsubsection{Главне разлике, истраживања, илустрација примером}
\subsection{Фундаментални проблем}
    \subsubsection{Рубинов каузални модел}

Фундаментални проблем каузалног закључивања, како га је назвао Пол Холанд 1986. године, чињеница да за сваку појединачну јединицу (особу, објекат) која је 
предмет истраживања може бити посматран само један од два потенцијална исхода. То значи да не можемо истовремено посматрати шта се дешава са једним појединцем након 
што је примио третман (нпр. $T=1$) и шта се дешава са тим истим појединцем након што није примио третман (нпр. $T=0$), у истом тренутку. Другим речима, каузални ефекат 
на индивидуалном нивоу увек остаје делимично непознат, јер је један од исхода нужно контрафактуалан (непосматран, хипотетички исход, супротан посматраним чињеницама).
Због ове немогућности истовременог посматрања, никада не можемо директно измерити индивидуални каузални ефекат.

Доналд Рубин је 1974. године формализовао овај проблем и увео основни језик за кодирање каузалности. Овај модел назива се Рубинов Каузални Модел (РЦМ), састоји се 
из три градивна блока, кључних за дефинисање каузалног ефекта.
\begin{enumerate}
    \item \textbf{Потенцијални исходи} За сваку јединку $i$, дефинишу се два потенцијална исхода, $Y_i^1$ или $Y_i^0$ у зависности од тога да ли 
    је јединка примила третман ($T_i=1$) или није ($T_i=0$). Пре доношења одлуке о додели третмана, оба ова стања су теоретски изводљива за сваку јединку па се каузални 
    ефекат дефинише као разлика потенцијалних исхода, односно $\Delta^Y_i = Y_i^1 - Y_i^0$.
    \item \textbf{Правило доделе третмана} То је променљива $D_i$ која одлучује ко прима третман ($Т_i=1$) и ко остаје у контролној групи ($Т_i=0$). Правило 
    доделе је кључно јер одређује који од два потенцијална исхода ће бити посматран. Механизам додељивања третмана је важан јер утиче на очекиване вредности потенцијалних исхода.
    \item \textbf{Једначина преклапања, прекидна једначина} Ова једначина повезује посматрани исход $Y_i$ са потенцијалним исходима, преко променљиве $D_i$: $Y_i = D_i Y^1_i + (1 - D_i) Y^0_i$.
    Оне објашњава да посматрамо само исход који одговара третману који је заиста примљен.
\end{enumerate}
РЦМ је, у суштини, модел о делимично посматраним случајним варијаблама па се каузално закључивање може посматрати као предвиђање шта би се догодило са јединицом $i$ 
ако би $D_i=0$ или $D_i=1$.

Да би потенцијални исходи били добро дефинисани, неопходна је претпоставка о стабилној вредности третмана јединице (\textit{eng. Stable Unit Treatment Value Assumption, SUTVA}), 
која подразумева да исход једне јединке не зависи од третмана других и да не постоје скривене варијанте третмана. 
Na primer, ishod jedne osobe koja prima transplantaciju srca ne bi trebalo da zavisi od toga da li je i druga osoba primila transplantaciju.
npr. različiti hirurzi, procedure, oprema

Како индивидуални ефекти нису директно доступни, у пракси се користе користе агрегатне мере каузалног ефекта, попут просечног ефекта третмана 
(\textit{eng. Average Treatment Effect, ATE}), $\theta = E{Y_i(1)} - E{Y_i(0)}$, или просечног ефекта на третиране (\textit{eng. Average Treatment Effect on the Treated, ATT}), 
$\Delta^Y_{TT} = E{Y_i^1 - Y_i^0 | D_i=1}$. Ове мере омогућавају поређење група и дају емпиријски смисао истраживањима у којима се тражи процена узрочних утицаја. Да би 
се ови просечни ефекти могли проценити, поред СУТВА, потребне је и кључна статистичка претпоставка о јакој игнорабилности, која укључује:
\begin{itemize}
    \item \textbf{Неконфузност} Потенцијални исходи су независни од доделе третмана, условљено коваријатама\footnote{Предтретмантске контролне променљиве $X_i$ су све оне које су посматране или мерене 
пре него што је јединка примила третман. Служе као улазни подаци у регресионим моделима ради постизања условне заменљивости између третиране и контролне групе.}:
$$
(Y^1_i, Y^0_i) \perp D_i | X_i
$$
Ово значи да је, унутар слојева дефинисаних коваријатама, третман насумично додељен. Неконфузност је од суштинског значаја за елиминисање пристрасности услед селекције и посебно 
је важна у опсервационим студијама.
    \item \textbf{Позитивност} За сваку комбинацију вредности коваријата $X$, постоји ненула вероватноћа да ће јединица примити и третман и контролу:
$$
0 < Pr(D_i=1|X_i=x) < 1
$$
У случају кршења ове претпоставке, просечан каузални ефекат не може бити процењен ни са бесконачном количином података. Позитивност је кључна за методе засноване на 
вероватноћи третмана(\textit{eng. Propensity Score}), јер се условна средња вредност не може дефинисати ако је вероватноћа примања третмана, за дату комбинацију коваријата, једнака нули.
\end{itemize}

У практичном контексту, ATE омогућава да проценимо колики би био очекивани ефекат третмана када би се применио 
на целу циљну групу, јер се у друштвеним, биомедицинским и економским истраживањима ретко усредсређујемо на индивидуалне ефекте. 
Уместо тога, истраживаче углавном занима просечан утицај третмана на популацији, што има директне импликације за 
креирање јавних политика, медицинске интервенције или економске мере.

Ипак, ATE има и одређена унутрашња ограничења која је важно разумети приликом њене интерпретације:
\begin{enumerate}
    \item \textbf{Неидентификованост индивидуалних ефеката:}
Како индивидуални ефекти нису директно доступни, ATE се односи на просечан каузални ефекат у популацији, а не на појединачне јединице.
    \item \textbf{Прикривање хетерогености ефеката:}
Како АТЕ представља просек ефекта у популацији, она може прикрити значајне разлике међу појединцима или подгрупама. На пример, ако третман једнако помаже 
једној половини популације, а одмаже другој, просечни ефекат може бити нула, иако су индивидуални ефекти изражени и хетерогени.
    \item \textbf{Зависност од карактеристика популације:}  
Величина ATE-а зависи од дистрибуције индивидуалних ефеката у посматраној популацији. Ако одређени фактори (нпр. пол, старост или социоекономски статус) модификују 
ефекат третмана, ATE ће се разликовати између популација са различитом структуром ових фактора. Стога, ова мера није нужно стабилна при преношењу резултата на друге 
популације.
    \item \textbf{Зависност од претпоставки и механизма доделе третмана:}  
Процена ATE-а у великој мери почива на претпоставкама о начину доделе третмана и о одсуству конфаундерa. Уколико третман није насумично додељен и ако нису 
задовољене кључне претпоставке попут неконфузности и позитивности, процене ATE-а могу бити пристрасне.
    \item \textbf{Предиктивни, а не објашњавајући карактер:}  
Каузално закључивање се може посматрати као предвиђање онога што би се догодило јединки $i$ под различитим условима третмана, стога ATE даје просечну предикцију за 
целу популацију.
\end{enumerate}

С обзиром на наведена ограничења, процена ATE-а у пракси захтева пажљиво осмишљене статистичке приступе који омогућавају да се „реконструишу” недостајући 
контрафактуални исходи. Циљ ових метода је да, под одређеним претпоставкама, омогуће непристрасну и 
поуздану процену просечних ефеката третмана. У зависности од контекста истраживања и начина доделе третмана, користе се различите стратегије као што су 
регресиони естиматори, методе засноване на вероватноћи третмана, метода упаривања јединица са сличним карактеристикама. Све ове методе имају 
заједнички циљ: да обезбеде што бољу апроксимацију контрафактуалних исхода и тиме омогуће валидно каузално закључивање.

Процена ATE-а и илустрација приближавања контрафактуалних исхода указују на ограничења једноставних просечних ефеката и потребу за методама које контролишу доделу 
третмана и конфаундере. У наредним деловима рада разматраћемо рандомизоване експерименте, где насумична додела минимизује проблем контрафактуалности, и опсервационе 
студије, које се ослањају на статистичке приступе за валидну процену каузалних ефеката.

Тако дефинисано, процена ATE-а представља само један део ширег оквира каузалног закључивања. У наредним деловима рада фокусираћемо се на конкретне приступе који се 
користе за процену каузалних ефеката у различитим контекстима: прво ћемо размотрити рандомизоване експерименте, који омогућавају директну контролу доделе третмана и 
минимизирају проблем контрафактуалности, а затим ћемо се осврнути на опсервационе студије, у којима се процене ATE-а ослањају на статистичке методе и претпоставке за 
контролу конфаундера.

    \subsubsection{АТЕ (Average Treatment Effect), дефиниција}
    \subsubsection{проблем контрафактуала, немогућност да посматрамо оба света истовремено}
\subsection{Математичке основе иза концепта узрочности}
    \subsubsection{АТЕ (процена и математичке импликације)}
    \subsubsection{СУТВА}
    \subsubsection{алгоритми за процену ефекта}
\subsection{Утврђивање кауланости и експерименти (прелазак на следеће поглавље)}

\newpage



\section{Рандомизирани експеримент}
\newpage



\section{Опсервационе студије}


\end{document}

