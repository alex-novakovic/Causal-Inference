\documentclass[12pt, a4paper]{article}

\usepackage{authblk}
\usepackage[utf8]{inputenc}
\usepackage[T2A]{fontenc}
\usepackage[serbianc]{babel}
\usepackage{hyperref}

\renewcommand\Authsep{\par}
\renewcommand\Authands{\par}

\title{Увод у каузално (узрочно) закључивање}
\author{Александра Новаковић 368/22}
\author{Милица Ињац 338/18}
\author{Бојан Корда 121/19}
\affil{Математички факултет, Универзитет у Београду}
\date{\today}

\begin{document}
\maketitle
\newpage

\tableofcontents
\newpage

\section{Фундаментални проблем}

- Увод
    - корелација не повлачи узрочност (пр. продаја сладоледа и бр дављења на плажи)
    - зашто је важно (медицина, друштвене науке, економија)
    - Главне разлике, истраживања, илустрација примером 
- Фундаментални проблем 
    - Рубинов каузални модел
    - АТЕ (Average Treatment Effect), дефиниција
    - проблем контрафактуала, немогућност да посматрамо оба света истовремено
- Математичке основе иза концепта узрочности
    - АТЕ (процена и математичке импликације)
    - СУТВА
    - алгоритми за процену ефекта
- Утврђивање кауланости и експерименти (прелазак на следеће поглавље) 

\newpage



\section{Рандомизирани експеримент}
\newpage



\section{Опсервационе студије}


\end{document}

