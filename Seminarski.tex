\documentclass[12pt, a4paper]{article}

\usepackage{authblk}
\usepackage[utf8]{inputenc}
\usepackage[T2A]{fontenc}
\usepackage[serbianc]{babel}
\usepackage{hyperref}
\usepackage{amsmath}

\renewcommand\Authsep{\par}
\renewcommand\Authands{\par}

\title{Увод у каузално (узрочно) закључивање}
\author{Александра Новаковић 368/22}
\author{Милица Ињац 338/18}
\author{Бојан Корда 121/19}
\affil{Математички факултет, Универзитет у Београду}
\date{\today}

\begin{document}
\maketitle
\newpage

\tableofcontents
\newpage

\section{Фундаментални проблем}
\newpage



\section{Рандомизирани експеримент}
\newpage



\section{Опсервационе студије}
\subsection{Увод}
За разлику од рандомизираног експеримента, где се третман насумично додељује међу изабраним јединицама или се насумично бира скуп јединица за сваку врсту третмана, опсервационе студије су студије у којима истраживач посматра шта се дешава у реалном свету, без контролисања или мењања услова. У односу на рандомизирани експеримент, опсервационе студије имају неколико предности и мана. Њихова предност огледа се у широм спектру популације који можемо посматрати, уштеди средстава која су потребна за рандомизирани експеримент, као и већој екстерној валидности с обзиром да опсервационе студије показују како третмани или понашања заиста утичу на људе у свакодневном животу. Међутим, мана ове врсте студија је сама чињеница да додела третмана није одабрана као у претходном случају, већ постоје многобројни додатни утицаји како на исход, тако и на саму доделу третмана. Њих називамо конфаудерима или сметајућим коваријатама. 

Дакле, опсервационе студије представљају истраживања у којима механизам доделе третмана није под контролом истраживача, али га можемо одређеним методама проценити регуларним, што ће омогућити да донесемо закључке о каузалног ефекту налик процени у оквиру рандомизираног експеримента. То постижемо различитим алатима којима ћемо контролисати коваријате. Циљ контролисања коваријата је да унутар субпопулација дефинисаних истим вредностима коваријата, додела третмана буде независна од потенцијалних исхода, како би се, пре свега елиминисали ефекти конфаундера који би могли утицати на сам ефекта третмана.

Један релевантан пример би била студија где желимо да проценимо ефекат редовног вежбања на здравље срца. Међутим, уколико у обзир не узмемо старост посматраног појединца, нећемо добити јасну процену. Због тога, старост и остале конфаудере морамо да контролишемо, тј. урачунамо њихов утицај. 
Када је реч о опсервационој студији разликујемо модел-базирани приступ где тежимо да имплементирамо што вернији модел који описује податке, а коефицијент уз третманску променљиву предтсвља вредност каузалног ефекта. Други, углавном прецизнији приступ је дизајн-базирани. Такве опсервационе студије се могу поделити у две кључне фазе, фазу дизајна и анализе. У оквиру фазе дизајна, прелиминарном анализом података о коваријатама и додели третмана, а без употребе података о исходу, конструишемо узорак или план анализе који обезбеђује што бољи баланс коваријата између третиране и контролне групе. Овај приступ је паралелан фази дизајна у стратификованим рандомизованим експериментима, где је баланс гарантован самим дизајном. То се, у случају опсервационих студија, постиже на различите начине и помоћу разних алата. 
Важно је нагласити да се најбоље процене добијају комбиновањем оба приступа.
\subsubsection{Кључни алати и коцепти у фази дизајна}
Фаза дизајна фокусира се искључиво на податке о додели третмана и преретманским коваријатама, док се подаци о исходу се не користе. Дизајнирање опсервационе студије без увида у податке о исходу је кључно јер спречава истраживача да свесно или несвесно прилагоди модел како би одговарао унапред замишљеним резултатима. Ова фаза обухвата неколико повезаних корака којима се имплементира концепт балансирајућих мера, од којих је најважнија вероватноћа доделе третмана. Балансирајућа мера је функција коваријата која има својство да, када се на њу услови, вероватноћа добијања активног третмана постаје независна од самих коваријата. Вероватноћа доделе третмана, u ознаци $e(x)$, представља посебну врсту балансирајуће мере, дефинисана као условна вероватноћа пријема третмана условљена коваријатама.  

Кључни део фазе дизајна је процена степена баланса у дистрибуцијама коваријата између третиране и контролне групе. Идеално би било када би коваријате имале сличне вредности у обе групе, али с обзиром да то веома често није случај, када проценимо разлике међу њима користимо разне методе како би се тај баланс побољшао. За почетак  процењујемо вероватноћу доделе која заправо представља вероватноћу да одређена јединица прими третман, да бисмо његовом применом могли да наставимо са дизајном. Веома важну улогу у овој фази има  Matching (упаривање) које спроводимо тако да за сваку третирану јединицу проналазимо одговарајућу контролну у смислу вероватноће доделе или Махаланобисове метрике. Веома је значајан и  Trimming (одбацивање), поступак где се одбацују јединице за које уочимо недостатак преклапања. Недостатак преклапања указује на то да јединице са супротним третманом имају веома различите вредности коваријата, што чини екстраполацију нетачном и осетљивом. Иако се фаза дизајна првенствено бави балансом, она такође укључује прелиминарне анализе за процену плуазибилности кључне, али нетестибилне, претпоставке неконфузности (Unconfoundness).
\subsubsection{Кључни алати и концепти у фази анализе}
Када смо завршили са фазом дизајна, податке који су у оквиру те фазе припремљени и балансирани узимамо и примењујемо различите технике за коначну, валидну процену каузалног ефекта. Јенда од најзначајнијих метода је стратификација, којом податке делимо на страте. Страте су подкласе формиране на основу вероватноће доделе третмана, тако да су у сваком страту те вероватноће приближније него када посматрамо цео узорак. У идеалном случају, коваријате су избалансиране и студију  третирамо као да је спроведен стратификовани рандомизовани експеримент. Даље, просечни ефекат третмана се може проценити као пондерисани просек разлика у просечним исходима између третиране и контролне групе унутар сваког стратума. Због тога је веома значајна тежинска регресија. Пондери су углавном засновани на величини подкласе. Такође, стратификација се често комбинује са другим методама ради додатног прилагођавања за евентуалне преостале разлике у коваријатама унутар слојева. Регресија је један од алата, који се може користити за прилагођавање за преостали небеланс коваријата, чиме процене чини прецизнијим. Модел базирана анализа користи регресиони модел како би импутирала недостајуће потенцијалне исходе за третиране јединице и на тај начин решавамо преостале мале разлике у коваријатама унутар парова.
\subsection{Дизајн фаза}
У овом делу разлажемо неопходне претпоставке и концепте који морају бити успостављени пре анализе исхода. 
\subsubsection{Игнорабилност и регуларни механизми доделе}
Раније смо већ поменули регуларне механизме доделе и напоменули да уколико можемо проценити нашу доделу третмана регуларном, онда можемо проценити и каузални ефекат. Јако је важно да можемо да претпоставимо да се ради о оваквој додели, јер у супртоном каузална интерпретација постаје веома нестабилна. Када се ради о рандомизованом експерименту, сам његов дизајн обезбеђује поменуту претпоставку. Када је реч о опсервационим студијама, разматрамо три кључна сегмента која заједничким именом називамо игнорабилност и под чијим условом можемо сматрати да важи регуларни механизам доделе. 
Формално, игнорабилност је независност потенцијалних исхода од доделе третмана условљеном коваријатама: $(Y(0), Y(1)) \perp T \mid X$. Регуларни механизам доделе је начин доделе третмана који је предвидив и зависи само од познатих карактеристика, што омогућава валидно поређење група у опсервационој студији.

Да би игнорабилност била испуњена треба да важе три услова.
\begin{itemize}
  \item1.SUTVА ( индивидуалистичка додела)

Третман примењен на једну јединицу не утиче на исход за друге јединице. 

Јединица која прима специфични ниво третмана не може примити различите облике тог третмана.
  \item 2.Услов позитивности (пробабилистичка додела)
Механизам доделе $\Pr(T \mid X, Y(0), Y(1))$ је пробабилистички ако је вероватноћа доделе третмана за јединицу $i$ строго између нуле и један тј. свака јединица има могућност да буде додељена и активном третману и контролном третману
  \item 3. Неконфузност (незбуњивост)
Механизам доделе је незбуњив уколико не зависи од потенцијалних исхода: $\text{P}(T \mid X, Y(0), Y(1)) =`\text{P}(T \mid X, Y'(0), Y'(1))$ за све $T, X, Y(0), Y(1), Y'(0)$ и $Y'(1)$, где су $Y(0), Y(1), Y'(0), Y'(1)$ потенцијални исходи.
Ова претпоставка је нетестибилна и врло вероватно на самом почетку није испуњена, али коришћењем различитих метода можемо је учинити плаузибилном. То постижемо прилагођавањем (контролом) свих посматраних коваријата. Ако је механизам незбуњив, можемо избацити потенцијалне исходе као аргументе и писати механизам додељивања као  $P(Т \mid X)$. 
\end{itemize}
\subsubsection{Вероватноће доделе третмана}
Горе поменути услов позитивности захтева да вероватноћа доделе третмана $e(x) = \text{P}(Т_i = 1 \mid X_i = x) = E[Т_i \mid X_i = x]$ који је формално дефинисан као условна вероватноћа пријема третмана $$(Т_i=1)$$, буде строго већи од нуле и строго мањи од јединице.  У опсервационим студијама, ова вероватноћа готово никада није позната, већ се мора проценити, обично коришћењем логистичке регресије. Спецификација поменутог модела може укључивати само линеарну везу коваријата, а можемо по потреби додати и квадратне и интеракционе термине.  Битно је наглсаити да циљ није добити најтачнију оцену хипотетички истинске вероватноће доделе третмана, већ пронаћи спецификацију која постиже адекватан баланс коваријата у узорку. 
Баланс се проверава након што се узорак стратификује према процењеној вероватноћи доделе, тако што се проверава да ли су просеци коваријата слични између третиране и контролне групе унутар тих подкласа. О томе ћемо више говорити у фази анализе.

Применом регресије $\text{logit}(\P(Т_i = 1 \mid X_i)) = \beta_0 + \beta_1 X_{i1} + \dots + \beta_k X_{ik}$ добијамо процену вероватноће да јединица i добије третман $e(X_i) = \P(Т_i = 1 \mid X_i)$. На овај начин, све коваријате смо сложили у једну скаларну врендост што је од изузетног значаја за даљу анализу.
\subsubsection{Упаривање}
Упаривање се заснива на налажењу директних поређења за сваку јединицу.  Примарни циљ упаривања је да креирамо под-узорак јединица у којем су дистрибуције коваријата између третиране и контролне групе боље избалансиране. За дату третирану јединицу са одређеним скупом вредности коваријата тражимо контролну јединицу са што сличнијим скупом коваријата. На пример, претпоставимо да желимо да проценимо ефекат програма обуке за посао на исходе на тржишту рада за одређену особу, рецимо, тридесетогодишњу жену са троје деце млађе од 10 година, са завршеним факултетом и три године радног искуства, која је прошла кроз програм обуке. У овом случају коваријатама сматрамо број деце, њихову старост, ниво образовања и године искуства. У приступу упаривања циљ је да пронађемо особу из контролне групе која има исте вредности коваријата. Међутим, уколико не можемо пронаћи такву јединицу овај приступ постаје теже изводљив. Можемо тражити старију жену али са више радног искуства или млађу жену са мање деце и мање радног искуства. Балансирајућа мера, попут вероватноће доделе третмана решава овај проблем. Дакле, није неопходно да све коваријате имају исте вредности, већ да буду избалансиране. Да бисмо успешно применили упаривање, неопходна је метрика удаљености, која нам омогућава да проценимо колико су две јединице сличне. Постоји неколико начина да израчунамо тражену метрику, а најчешће су следеће:
\begin{itemize}
  \item  Метрика за упаривање на основу линеаризоване вероватноће доделе:
Процена линеаризована вероватноће је $\hat \ell(x) = \ln\left(\frac{\hat e(x)}{1 - \hat e(x)}\right)$, где је $\hat e(x) = \widehat{\P}(T = 1 \mid X = x)$ процена вероватноће да јединица са коваријатама $x$ добије третман, па је метрика рачуната у односу на њега дата формулом $d_{\hat \ell}(x, x') = (\hat \ell(x) - \hat \ell(x'))^2$, где су $x$ i $x'$ вектори коваријата за јединице којима процењујемо сличност.
  \item  Махаланобисова метрика: $d_{\text{Mahal}}(x, x') = \sqrt{(x - x') \hat{V}^{-1} (x - x')^\top}$ где је  ${\text{mahal}}$ коваријатна матрица коваријата x, која узима у обзир варијансу и корелацију између свих коваријата. Када се користи Махаланобисова метрика за упаривање, матрица тежине $V_{\text{mahal}}$ се обично заснива на просеку матрица коваријансе унутар третиране ($Т_i=1$) и контролне ($Т_i=0$) групе:

$V_{\text{mahal}} = \frac{1}{2} \left( \hat{\Sigma}_c + \hat{\Sigma}_t \right)$ где су  $\hat{\Sigma}_c$ (или $S_c$) узорочна матрица коваријансе коваријата у контролној групи  $\hat{\Sigma}_t$ (или $S_t$) узорочна матрица коваријансе коваријата у третираној групи.
\end{itemize}
Затим, када имамо метрику, постоји више принципа упаривања, а кључна стратегија је у погледу тога да ли конролне јединице могу бити коришћене више пута. Тако, имамо могућност:
\begin{itemize}
  \item Упаривање без понављања

Свака контролна јединица се може користити само једном као пар третираној јединици. Овде се користи секвенцијални алгоритам упаривања. Јединице се обично сортирају на неки начин, а затим се секвенцијално упарују, почевши од јединица које је најтеже упарити. Сортирање се обично врши по опадајућој вредности вероватноће доделе третмана и тим редом се и упарују.
  \item Упаривање са понављањем

Контролна јединица може бити употребљена као најбољи пар више пута за различите третиране јединице. Ова метода олакшава рачунање и побољшава квалитет упаривања јер једна контролна јединица може послужити као најбољи пар за више сличних третираних јединица. Такође, резултат не зависи од редоследа упаривања. Међутим, овде се дешавају компликације у процени варијансе због зависности између парова.
\end{itemize}
Иако смо до сада често говорили о просечном ефекту третмана (АТЕ), упаривање је, због своје природе, првенствено усмерено на процену другог естиманда, тј. просечног ефекта третмана за третиране јединице (ATT). Упаривање обично настоји да пронађе што сличније контролне јединице за сваку третирану. Ово резултира скупом података који је релевантан за процену каузалног ефекта само за подпопулацију третираних јединица. Процедуре упаривања могу се проширити како би се проценио и укупни АТЕ (просечан ефекат третмана за цео узорак) тако што би се упаривање применило и на контролне јединице, али је то у већини случајева непрактично, тако да користимо друге методе како бисмо на крају заиста добили просечни ефекат третмана за цео узорак. Упаривање је ретко савршено, јер увек постоји извесна неусклађеност у коваријатама, између третиране јединице и њеног пара.Ова пристрасност настаје јер потенцијални исход контролне јединице$ Y_i(0)$није савршен сурогат за недостајући потенцијални исход третиране јединице  $Y_i(1)$ , управо услед преостале разлике у коваријатама. Постоје поступци прилагођавања пристрасности  којима се овај ефекат може ублажити, најчешће применом линеарне регресије, али се могу применити у те сврхе и стратификација или пондерисање који у значајној мери редукују преосталу пристрасност.

\subsubsection{Одбацивање}
Повезано са упаривањем је и коришћење технике одбацивања лоших парова. Уколико је најближи пар за третирану јединицу и даље значајно различит (дистанца прелази одређену границу), та третирана јединица се може избацити из анализе. Одбацивање јединица за које не постоји одговарајућа у супротној третманској групи осигурава да закључци буду изведени само за подпопулацију где постоји довољно преклапања, чиме се повећава интерна валидност и кредибилитет.  За јединице са вредностима коваријата таквим да је вероватноћа добијања третмана близу нуле или један, тешко је добити прецизне процене типичног ефекта третмана. Уколико је вероватноћа да јединица прими третман близу јединици, биће више третираних у односу на контролне јединице, и обрнуто, уколико је поменута вероватноћа блиска нули. Тада, поређења захтевају екстраполацију, чинећи закључке непоузданим. Из тог разлога, одбацујемо јединице са екстремним вредностима поменуте вероватноће, и на тај начин мењамо естиманд, добијамо АТТ уместо АТЕ.  Интервал за вероватноћу третмана који се поставља је облика $(\alpha, 1-\alpha)$ и на тај начин добијамо поузданији резултат. Дакле, скуп јединица који посматрамо може се записати на следећи начин:  $C^* = \{ x \in X \mid \alpha \le e(x) \le 1 - \alpha \} \quad$. 

Међутим, потребно је на прави начин одредити $\alpha$.  Алгоритам прво проверава да ли је  задржавање целог узорка, $C=X$, могуће и статистички пожељно. То се проверава неједнакошћу која укључује просечну вредност инверзне варијансе вероватноће третмана $$\frac{1}{N} \sum_{i=1}^{N} \hat{e}(X_i) \cdot (1 - \hat{e}(X_i)) \le 2$$, где је N величина узорка. Уколико ова неједнакост не важи, то значи да екстремне вредности е(x) превише утичу на варијансу, па је потребно одбацивање јединица. У том случају,$ \gamma$ се дефинише као највећа вредност која задовољава услов

 $$\frac{1}{N} \sum_{i=1}^{N} \hat{e}(X_i) \cdot (1 - \hat{e}(X_i)) \cdot \mathbf{1}\{ \hat{e}(X_i)(1 - \hat{e}(X_i)) \ge \frac{1}{\gamma} \}= \gamma \cdot \frac{1}{N} \sum_{i=1}^{N} \mathbf{1}\{ \hat{e}(X_i)(1 - \hat{e}(X_i)) \ge \frac{1}{\gamma} \}$$

Тада је оптимална граница $\alpha$ дата на следећи начин:\\$$\alpha = \frac{1}{2} - \sqrt{\frac{1}{4} + \frac{1}{\gamma}}$$
Коначно, избацују се све јединице, чија је вероватноћа доделе третмана изван интервала  $(\alpha, 1-\alpha)$.
\subsection{Фаза анализе}
\subsubsection{Стратификација}
Стратификација је водећи приступ у анализи података са регуларним механизмом доделе третмана и користи се за естимацију каузалних ефеката након што је већ извршено потенцијално тримовање у фази дизајна. 
Будући да у опсервационим студијама обично постоји превише коваријата које узимају превише различитих вредности, то чини директну стратификацију на основу свих коваријата неизводљивом. Због тога, узорак се дели на подкласе тј. стратуме применом процењене вероватноће доделе $ \hat{e}(X_i)$, која је заправо скаларна функција коваријата. Иако та верововатноћа није директна коваријата, она је довољна да се, условно на њену вредност, елиминише пристрасност повезана са разликама у свим осталим коваријатама.

Узорак  делимо на стратуме, тако да је унутар сваког страта процењена вероватноћа доделе приближно константна.  Након поделе, податке анализирамо као да су настали из стратификованог рандомизованог експеримента, где се додела третмана сматра потпуно насумичном унутар сваке подкласе. Цео узорак можемо поделити на произвољан број страта, за који оценимо да би био довољан, а уколико желимо изузетну прецизност, та подела се спроводи на основу вредности већ поменуте линеаризоване вероватноће доделе.

$$\ell(X_i) = \ln \left( \frac{1 - \hat{e}(X_i)}{\hat{e}(X_i)} \right)$$

Дефинишемо интервал који је подскуп интервала који смо користили за одбацивање па самим тим представља подскуп јединица које имају довољан преклоп.
Границе интервала $b_0 \quad \text{и} \quad b_J$ постављају се на основу најмањих и највећих процењених вероватноћа доделе третмана у обе групе.  За доњу границу узимамо најмању вредност те вероватноће међу третираним јединицама, $$\hat{e}_c = b_0 = \min_{i: D_i = 1} \hat{e}(X_i)$$, док за горњу границу узимамо његову највећу вредност међу контролним јединицама $$\hat{e}_t = b_J = \max_{i: D_i = 0} \hat{e}(X_i)$$. Овај потез осигурава да у преосталом узорку не постоје јединице за које нема поређења у супротној групи. Алгоритам за стратификацију затим креће са једним блоком (J=1) дефинисаним претходним интервалом. Унутрашње границе bj проналазе се кроз процес у коме се постојећи, неадекватно балансирани блокови деле на мање, све док унутрашња варијација вероватноће доделе не постане прихватљиво мала.

\subsubsection{Модел базирани приступ}
Као што смо у уводу поменули, модел базирани приступ може бити опсервациона студија за себе, а може се и комбиновати са дизајн-базираном опсервационом студијом. Уколико комбинујемо ова два приступа, логичан потез после стратификације би било коришћење модела у оквиру сваког стратума. Пожељно је додатно побољшати прецизност коришћењем линеарне регресије, с обзиром да је вероватноћа доделе у блоковима приближна међу јединицама, а не потпуно једнака.  Модел-базирано прилагођавање унутар ових блокова служи за додатно смањење преостале пристрасности и обезбеђује највећу могућу прецизност.  У пракси се често, након стратификације, модел-базирано прилагођавање спроводи тако што се унутар сваког стратума j процењује регресиони модел $Y^{\text{obs}} = \alpha(j) + \tau(j) \cdot D_i + X_i \beta(j) + \epsilon_i$,  где је $Y_i^{\text{obs}}$ је посматрани исход, $X_i\beta(j)$ представља линеарно прилагођавање коваријатама унутар блока, а параметри $\alpha(j), \tau(j), \beta(j)$ се оцењују искључиво на основу јединица унутар блока.

Овај приступ омогућава да се унутар блока коригују преостале разлике у коваријатама, слично регресијском прилагођавању у рандомизованим експериментима, и резултује стратификационим оцењивачем са додатним прилагођавањем $\hat{\tau}^{\text{strat, adj}}$.
\subsubsection{Пондерисање}
\par Пондерисање, тачније увођење тежинa је једна од кључних метода анализе која се користи за превазилажење пристрасности у опсервационим студијама, где не постоји гаранција да су групе за третман и контролу упоредиве као у рандомизованим експериментима. Пондери, тј. тежине у регресији уско су повезане са е(x), коју користимо за њихово израчунавање тако да ефективно ребалансирају дистрибуцију коваријата између третиране и контролне групе. Циљ је да се креира виртуелни узорак у којем је додела третмана неконфундирана, омогућавајући непристрасну оцену ефекта третмана.

Први корак је рачунање пондера. 

Уводимо Ховиц томпсонове тежине: 

Тежина $A_i^{ht}$, за сваку јединицу i рачуна се као инверзна вероватноћа њеног стварног примања третмана или контроле, условљена њеним коваријатама па би формуле за третирану и котрнолни јединицу би изгледале овако:

\[
A_i^{ht} =
\begin{cases}
\displaystyle \frac{1}{\hat{e}(X_i)}, & \text{ако } T_i = 1 \text{ (третирана јединица)},\\[2mm]
\displaystyle \frac{1}{1 - \hat{e}(X_i)}, & \text{ако } T_i = 0 \text{ (контролна јединица)}.
\end{cases}
\]


На овај начин, јединици која је имала малу вероватноћу да буде третирана дајемо већу тежину, о обзиром да је она била сличнија контролној групи по карактеристикама, па тако на неки начин добијемо најбољу слику о томе шта би се десило да је јединица из контролне групе добила третман. Слично важи за јединице из контролне групе које су имале велику вероватноћу да приме третман.

Када смо израчунали пондер, даље, за рачунање просечног ефекта третмана, неизоставан је Ховиц-Томпсонов оцењивач, којим се рачуна просечан исход третираних и контролних јединица.  Теоријски, то су очекивања $E\left[ \frac{e(X_i)}{Т_i} Y_i^{obs} \right] = E_{sp}[Y_i(1)]$ за третирану јединицу  и $E\left[ \frac{1 - e(X_i)}{1 - Т_i} Y_i^{obs} \right] = E_{sp}[Y_i(0)]$ за контролну јединицу.  Да бисмо ове теоријске очекиване вредности претворили у вредности које се могу израчунати очекивања,а узорак обима N замењујемо емпиријским просецима:

\[
\hat{Y}(1) = \frac{1}{N_1} \sum_{i=1}^{N} \frac{T_i \, Y_i^{\text{obs}}}{\hat{e}(X_i)}, \quad 
\hat{Y}(0) = \frac{1}{N_0} \sum_{i=1}^{N} \frac{(1 - T_i) \, Y_i^{\text{obs}}}{1 - \hat{e}(X_i)}
\]
, где је $N_1$ број јединица у третираној групи, а  $N_0$ број у контролној.
 
Коначна процена каузалног ефекта је 

$
\hat{\tau}^{HT} 
= \hat{Y}(1) - \hat{Y}(0)
= \frac{1}{N} \sum_{i=1}^{N} \frac{T_i \, Y_i^{\text{obs}}}{\hat{e}(X_i)}
- \frac{1}{N} \sum_{i=1}^{N} \frac{(1 - T_i) \, Y_i^{\text{obs}}}{1 - \hat{e}(X_i)}.
$

Најчешћи начин да имплементирамо Ховиц Томпсонов оцењивач у статистичком софтверу је кроз пондерисану регресију најмањих квадрата. Метода пондерисања је довољно флексибилна да се у оквиру исте регресије могу укључити и коваријате, што чини оцењивач двоструко робусним. Основна пондерисана регресија би изгледала овако $Y_i^{\text{obs}} = \alpha + \tau \cdot Т_i + \epsilon_i$.

Уколико смо претходно извели стратификацију и затим применили регресију унутар сваког блока $j$, добијамо унутар-блок оцењивач ефекта третмана $\hat{\tau}^{\text{adj}}(j)$.  
Тада, вршимо агрегацију ефекта и укупни просечан ефекат третмана за цео узорак добијамо пондерисањем по релативној величини блока $q(j) = \frac{N(j)}{N}$:
$\hat{\tau}^{\text{strat}} = \sum_{j=1}^{J} q(j) \cdot \hat{\tau}^{\text{adj}}(j)$ где $N(j)$ је број јединица у страту $j$, а $N$ укупан број јединица у узорку.  
На овај начин, пондери осигуравају да укупни процењени ефекат третмана одражава релативне величине страта.

\subsection{Пример}
   \subsubsection{О бази}
База података NHEFS је направљена током студије која је истраживала утицај клиничких, нутритивних и других фактора на здравље одраслих особа у Сједињенин Америчким државама. Подаци су прикупљени током прве Националне анкете о здрављу и исхрани између 1971. и 1982. године. Ми ћемо се бавити испитивањем утицаја престанка пушења на промену у килажи.
   \subsubsection{О коду}
За почетак, користићемо модел базирани приступ, а касније ћемо додатним техникама испитати каузални ефекат. У моделу fit претпоставили смо да постоји веза између интензитета престанка пушења са променом у тежини, тј. они који су интензивније пушили ће вероватно лакше добити на килажи када оставе пушење. Осим тога, контролисали смо све коваријате за које смо сматрали да су релевантне.  Укључили смо и квадратне термине где смо сматрали да је потребно, осим поменуте интеракције. Неко време ћемо се задржати на њеној процени. Да бисмо ову интеракцију додатно испитали, направили смо матрицу контраста за експлицитно израчунавање комбинација коефицијената из модела. Она нам омогућава рачунање ефекта третмана за специфичне вредности, у овом случају smokeintensity. На крају смо као процену ефекта пушења на промену тежине за интензитет од 5 цигарета дневно добили приближно 2.8kg, док је за интензиет од 40 цигарета дневно повећање тежине близу 4.4kg. Такође, на основу интервала поверења, можемо закључити да је ефекат статистички значајан, па на основу ових процена видимо да поменута интеракција има ефекта на циљну променљиву. Међутим, у практичном смислу нисмо уочили велику значајност. Због тога, креирамо модел без интеракционог термина, како бисмо додатно испитали да ли нам је он уопште неопходан. 

\par Поређењем два модела, на основу p-вредности, $\text{AIC}$ и $\text{BIC}$ мера квалитета модела доносимо закључак да је други модел бољи, иако имају приближне вредности. Ова процена нам говори да интеракциони термин није био значајанза модел, па га одбацујемо. 

Када смо проценили да је модел fit2 бољи, можемо закључити да је модел-базираним приступом добијени просечни каузални ефекат престанка пушења на тежину 3.5kg. Дакле, у просеку су се људи угојили 3.5kg када су престали да пуше.

Сада приступамо дизајн-базираном приступу, где ћемо коришћењем различитих алата покушати да сведемо опсервациону стдуију на студију сличну рандомизованој.

Први корак је, као што смо већ поменули рачунање вероватноће третмана. У те сврхе користимо логистичку регресију. Предуслов за успешно моделирање каузалног ефекта је да постоје преклапања, ондосно јединице из обе третманске групе са сличним вредностима вероватноче доделе. Први начин на који стичемо увид у потенцијална преклапања је приказом статистика оних који су престали да пуше и оних који нису, одакле видимо да је опсег вредности вероватноћа доделе сличан за обе групе, као и медијана и средња вредност, што указује на потенцијално постојање преклапања.  

Визуелни приказ преклапања је прецизнији, тако да цртамо плот густине расподеле који приказује расподелу процењених вероватноћа третмана, за обе третманске групе. Са плота се јасно види да постоји преклапање поменутих густина, што нам омогућава да поредимо особе са сличним особинама, али ипак видимо да постоје и неке разлике у карактеристикама између група које утичу на престанак пушења. Са следећег хистограма можемо да видимо колико конкретно има особа у свакој категорији вероватноће доделе. Уочавамо да нема преклапања у екстремним интервалима ове вероватноће, и због тога између осталог приступамо методи упаривања, како бисмо оне јединице третмана које нису упарене одбацили. 

Упаривање јединица вршимо уз помоћ поменуте logit метрике која користи линеаризовану вероватноћу доделе третмана, тако да се свакој третираној јединици додељује контролна која јој је најближа у смислу ове метрике. После упаривања, стандардизована разлика средњих вредности је мања за неке коваријате, као и разлика вероватноћа доделе, што пре упаривања није био случај. Као резултат добијамо боље изједначене третманске групе по свим коваријатама, тако да можемо добити непристрасну оцену ефекта третмана.
''''''''''''
Summary of Balance for All Data:
                                   Means Treated Means Control Std. Mean Diff. Var. Ratio
distance                                  0.3124        0.2450          0.5388     1.4114

Summary of Balance for Matched Data:
                                   Means Treated Means Control Std. Mean Diff. Var. Ratio
distance                                  0.3124        0.3071          0.0423     1.1829
''''''''''''
Када имамо нову базу са упареним подацима, над њом ћемо конструисати нови линеарни модел, а затим га упоредити са моделом који смо направили раније. Како бисмо оценили који модел нам даје поузданији каузални ефектат, разматраћемо неколико факотра:





На основу табеле примећујемо да модел са упареним подацима даје поузданију процену, међутим треба имати у виду да се каузални ефекат добијен на овај начин односи само на третиране јединице, тзб. АТТ.

Други приступ над упареним подацима је стратификација над упареним подацима.
















\end{document}




