\documentclass[12pt, a4paper]{article}

\usepackage{authblk}
\usepackage[utf8]{inputenc}
\usepackage[T2A]{fontenc}
\usepackage[serbianc]{babel}
\usepackage{hyperref}
\usepackage[amsthm]
\newtheorem{primer}[Пример]{section}

\renewcommand\Authsep{\par}
\renewcommand\Authands{\par}

\title{Увод у каузално (узрочно) закључивање}
\author{Александра Новаковић 368/22}
\author{Милица Ињац 338/18}
\author{Бојан Корда 121/19}
\affil{Математички факултет, Универзитет у Београду}
\date{\today}

\begin{document}
\maketitle
\newpage

\tableofcontents
\newpage

\section{Фундаментални проблем}
\newpage



\section{Рандомизирани експеримент}

У науци и друштвеним истраживањима је, баш као и у животу, веома битан концепт узрочности и 
узрочно-последичних веза. Важно је знати како неке промене утичу на конкретан појам и да ли га 
побољшавају, погоршавају или немају утицаја на њега. Како постоји могућност да су промене, 
заправо, последица неких других фактора, рандомизирани експерименти су јако корисни као 
методолошки оквир који омогућава најпоузданије одговоре на питања која се тичу узрочности.
//Први рандомизирани експерименти су се појавили првом половином 20. века и употребљавани су у 
медицини за тестирање ефикасности новооткривених лекова и терапија, а за једног од "твораца" 
рандомизираних експеримената се сматра Роналд Фишер. Од тада се њихова примена интензивно 
проширила и на друштвене науке, психологију, образовање, економију, као и на дигиталне технологије 
и интернет индустрију.

\subsection{Теоријска основа}


Рандомизирани експерименти се често описују помоћу Нојман-Рубин оквира потенцијалних исхода, 
а као што смо већ упознати, проблем који настаје због тога што никада за 
одређеног испитаника не можемо посматрати оба исхода већ само један се назива фундаментални 
проблем казуалног закључивања. У наставку ћемо сазнати како рандомизирани експерименти 
отклањају овај проблем. Кључ је у томе да третман и контрола буду додељени насумично јер су тада 
просечне вредности Y(1) и Y(0) у те две групе непристрасне оцене ефекта примања, односно, 
непримања третмана па се просечан ефекат третмана (АТЕ) може добити формулом ATE=E(Y(1)-Y(0)).
Рандомизирани експерименти на овај начин решавају проблем такозваног контрафактуалног питања које 
пита шта би се десило да нисмо применили третман. Како контролна група апроксимира контрафактуални 
сценарио третиране групе, у оквиру рандомизираног експеримента је одговор на ово питање јасан и 
не прави нам проблеме. Рандомизација обезбеђује да су стандардни статистички алати валидни и дају 
поуздане резултате па на основу података из оваквих експеримената можемо спровести тестирање 
хипотеза и израчунати р-вредност и интервале поверења. Као једина метода код које је осигурано да 
све разлике између експерименталне и контролне групе, рандомизација нам омогућава да се ефекат 
третмана изолује, као и да поређење група буде непристрасно, што је јако важно при коришћењу 
статистичких тестова попут АНОВА-е или t-теста који претпостављају да су јединице независне и 
идентично расподељене.
//Закључци које донесемо на основу резултата неког рандомизираног експеримента се могу односити на 
различите делове популације, у зависности од начина на који смо спроводили експеримент тј из које 
суперпопулације смо бирали јединке којима дајемо третман. Ако бисмо, на пример, из целе популације 
насумично бирали n$_0$ јединки за контролну групу и n$_1$ јединки које ће примити третман такође 
насумично изаберемо из целе популације, то би значило да се добијени резултати о ефективности 
третмана односе на целу популацију из које смо бирали јединке.
С друге стране, ако из читаве популације ненасумично одаберемо укупно n$_0$ + n$_1$ јединки и онда 
од њих насумично одређујемо којих n$_1$ ће примити третман онда се и добијени резултати односе на 
првобитно изабраних n$_0$ + n$_1$ јединки. Студије код којих су казуалне инференције оправдане 
само за одређени узорак или популацију су познате и као студије са интерном валидношћу, док се за 
оне чије се инференције могу генерализовати на ширу популацију каже да имају екстерну валидност.

\\subsubsection{Утицај пре-теста}

Иако су код рандомизираних експеримената јединице које примају третман насумично изабране, и даље 
постоји реална могућност да се оне у некој мери разликују, по вредности карактеристике која се 
испитује, од јединки које су сврстане у контролну групу. Уколико бисмо пре почетка примене третмана 
измерили ту вредност у обема групама, вредност коју бисмо добили тим мерењем би представљала 
пре-тест. Пре-тест би нам дао информацију о стартној разлици ижмеђу група, ако она постоји, а 
поред тога би и повећао прецизност добијених резултата јер би уврштавање пре-теста, као додатног 
предиктора, у регресиони модел отклонило претпоставку о томе да је једина промена која настаје 
последица примене третмана. Заиста, оваква претпоставка у већини случајева уопште није реална и 
може довести до, не тако занемарљивих, промена у резултатима. Ако, на пример, тестирамо утицај 
пијења козијег млека на висину деце, у односу на децу која пију кравље млеко, корисно је измерити 
висину деце пре него што почну да пију одређено млеко јер бисмо тако проверили да ли је висина у 
обе групе иста, али и колико су та деца порасла сама од себе јер ће и деца у контролној групи 
имати одређени напредак у висини, сама од себе. Сада када знамо да вредност пре-теста повећава 
прецизност се намеће питање који је модел најбоље узети, а најбољи пут да дођемо до одговора на 
њега је кроз пример. Напоменимо да, код рандомизираних експеримената, можемо користити и додатне 
предикторе све док су у питању предитори који немају никакву међусобну повезаност са примањем 
третмана. У наведеном примеру са козијим млеком би чак било и пожељно уврстити у модел информације 
попут старости деце или њихових генетских предиспозиција које се тичу висине, али никако не бисмо 
смели да имамо предикторе који зависе од самог третмана (нпр да ли је млеко домаће или купљено у 
супермаркету).

\begin(primer)
На примеру тестирања раста деце у зависности да ли су пила козије или кравље млеко ћемо узети 
четири, наизглед, слична модела и уочити кључне разлике међу њима.
Модели које ћемо тестирати су модел где пост-тест, односно испитивани исход, зависи само од 
третмана, затим, модел где исход зависи и од третмана и од пре-теста, као и модел где се као 
зависна променљива узима разлика пост-теста и пре-теста, а једини предиктор је третман и четврти 
модел - модел у ком се за пост-тест као предиктор, поред пре-теста и третмана узима и генетска 
предиспозиција за раст сваког детета.
У формалном запису, то би изгледало овако: $M_1: Y_i=a_0+a_1*T_i+\epsilon_i$ ; 
$M_2: Y_i=a_0+a_1*T_i+a_2*X_i+\epsilon_i$ ; $M_3: Y_i-X_i = a_0+a_1*T_i+\epsilon_i$ и 
$M_4: Y_i = a_0+a_1*T_i+a_2*X_i+a_3*g_i+\epsilon_i$ , при чему су Y$_i$ вредности пост-теста, 
односно висине деце након периода истраживања, X$_i$ вредности пре-теста тј измерене висине деце 
пре примене третмна, T$_i$ индиктори примања третмана (сваки T$_i$ узима вредност 1 ако је i-то 
дете пило козије млеко, а 0 ако је пило кравље), a$_i$ су реални коефицијенти, а $\epsilon_i$ су 
грешке. Тестирањем ова 4 модела, видимо да први модел нема толико велику прецизност као преостала 
три. Можемо приметити да је модел M$_3$ јако прецизан, али има већу варијансу од другог и 
четвртог модела којима је прецизност, такође, јако велика па како M$_2$ и M$_4$ имају довољно 
сличну вредност стварној, а мању варијансу од трећег модела можемо закључити да су M$_2$ и M$_4$ бољи 
модели од М$_3$. Оно што још мпжемо уочити је чињеница да је, код свих ових модела, ефекат 
третмана који нас заправо занима управо коефицијент који стоји уз третман (a$_1$). Заиста, за 
сва четири модела важи да се крајње висине два детета са истим резултатима пре-теста и истим 
генетским предиспозицијама разликују управо у вредности овог коефицијента.
\end{primer}

\\subsubsection[Веза између зависности и узрочности код рандомизираних експеримената]

Уколико се рандомизирани експеримент заиста одвија у савршеним условима који обухватају правилно 
спроведен процес рандомизације, спречавање систематског губитка једне групе и добро дефинисан и 
примењен третман онда можемо причати о еквиваленцији асоцијације и узрочности.
Асоцијација би, у овом значењу, представљала уопштење зависности двеју променљивих односно 
евентуалну везу која постоји између њих, што се може и формално записати и као "Између променљивих 
X и Y постоји асоцијација ако важи $P(Y|X) \neq P(Y)$ ."
Да бисмо могли без нејасноћа да објаснимо зашто је код рандомизираних експеримената, који 
испуњавају све неопходне услове, асоцијација тј зависност исто што и казуалност тј узрочност, 
поћићемо од чињенице да је код рандомизованих експеримената управо та произвољност избора 
заслужна за отклањање свих кофаундинга који би потенцијално довели до зависности, али сада ћемо 
се мало удубити како бисмо то и показали. Као прво, поћићемо од чињенице да је третман Т изабран 
потпуно насумично и као такав је независтан од осталих предиктора. Дакле, сваки предиктор X$_i$ 
који представља неку карактеристику популације ће имати исту расподелу и у третман групама и у 
контролној. Ову независност третмана и предиктора ћемо користити како би доказали да је 
$P(Y|do(t)) = P(Y|t)$ при чему је do(t) функција do-calculus која означава да је извршена 
интервенција и 'насилно" постаљена вредност t за третман T.
//-\underline{Доказ:} Нека је X довољан скуп за прилагођавање (sufficient adjustment set) који 
потенцијално може садржати и непосматране варијабле, али рандомизација и њих балансира.
Сада имамо: 
$$P(Y|do(t))=\sum_x P(Y|t,x) P(x)=\sum_x \frac{P(Y|t,x)*P(x)*P(t|x)}{P(t|x)}=
\sum_x \frac{P(Y,t,x)}{P(t|x)}$$ користимо да је Т независно од X па је 
$$P(Y|do(t))=\sum_x \frac{P(Y,t,x)}{P(t)}=\sum_x P(Y,x|t)=P(Y|t)$$
Овим смо доказали да је код рандомизираних експеримената узрочност еквивалентна асоцијацији.
Овај доказ је могао бити спроведен и много интуитивније и неформалније користећи својство 
размењивости, односно чињеницу да заменом контролне и третман групе очекивано дејство остаје 
непромењено (уколико бацањем новчића одређујемо које ће јединице примити третман, замена значења 
"писма" и "главе" не утиче на коначан исход експеримента).

\subsection{Дизајн и типови рандомизираног експеримента}

\subsubsection{Дизајн рандомизираног експеримента}

Најбитнија ствар код рандомизираног експеримента је да се насумично одаберу испитаници који ће 
примити третман. Преостали испитаници, који се налазе у контролној групи служе за поређење. 
Како би се побољшала прецизност добијених резултата њима је могуће дати \textit{placebo} и тако 
неутралисати психолошки ефекат. Рандомизирани експерименти код којих испитаници не знају да ли су 
у третман групи или у контролној групи се називају слепи, а они код којих ни истраживачи нису 
упознати ко је у којој групи се називају двоструко слепи. Пре самог почетка експеримента је 
потребно да истраживач прикупи неопходне биографске податке о испитаницима и да се упозна са 
етичким питањима у вези извршавања експеримента. Биографски подаци ће бити корисни за одређивање 
подобности сваког од испитаника да учествује у експерименту, као и да помогну истраживачу да 
донесе одлуку о томе који тип експеримента је најбоље спровести уколико ова одлука није већ од 
раније донешена. Неке од најпознатијих и најчешће употребљаваних рандомизација су једноставна, 
стратификована, блок, кластер, адаптивна, степенаста, повезана...
Код свих наведних врста рандомизираног експеримента поред испитивања ефекта дејства по једног 
фактора, подједнако је могуће вршити и рандомизиране експерименте у којима се проучава 
истовремено дејство два или више фактора. Такви експерименти се називају факторски рандомизирани 
експерименти. Предност оваквих експеримената је што нам могу уштедети на времену и на количини 
узорка чињеницом да нам приказују појединачне ефекте свих третмана који су истовремено примењивани, 
а поред тога нам дају и информације о међусобним интеракцијама између третмана. Ипак, код 
факторских експеримената број експерименталних услова експоненцијално расте са бројем фактора што 
повлачи потребу за већим обимом узорка, а само тумачење интеракција између фактора је често врло 
комплексно.

\subsubsection{Једноставна рандомизација}

Код једноставне рандомизације се испитаницима, независно, додељује особина да ли су у третман 
групи или контролној групи, са једнаким вероватноћама, еквивалентно принципу да бацање новчића 
одлучује да ли ће испитаник добити третман. На исти начин се код експеримената у којима испитујемо 
утицај више третмана испитаницима додељује третман који ће примити.
Предност овог типа рандомизације је што је лака за имплементацију и нема много правила и 
ограничења, али је мана што подела може бити неуравнотежена, поготово код узорака са мањим обимом.

\begin{primer}
Имамо 3 врсте ђубрива А, Б И Ц и испитујемо какав је њихов утицај на принос парадајза. 
Тестираћемо на начин да свака биљка добије један од ова три третмана и обухватићемо два случаја.
У првом ће свака од биљака имати подједнаке вероватноће (по 1/3) да буде нахрањена са било којим 
од три понуђена ђубрива. У другом случају ћемо приказати како изгледа када нам је потребно да 
вероватноће примене третмана нису међусобно једнаке. У нашем примеру то се може десити ако немамо 
подједнаке количине сваког од три ђубрива па желимо да вероватноће буду пропорционалне количинама 
које поседујемо. Урадићемо експеримент за обе могућности и упоредити добијене резултате.
\end{primer}

\subsubsection{Стратификована рандомизација}

Код стратификоване рандомизације се узорак прво, на основу неке особине као што је пол, старост и 
слично, подели на стратификоване групе па се на свакој тако добијеној групи (стратуму) засебно 
ради рандомизација. Овај тип може бити погодан за контролу неких битних карактеристика и смањује 
варијансу АТЕ, али постаје поприлично компликован уколико имамо велики број група у које је 
потребно поделити узорак. 

\begin{primer}
Желимо да испитамо утицај једног математичког онлајн курса на резултате на тесту из математике.
Имамо 800 испитаника различитих старосних доби и нивоа образовања и поделићемо их у четири групе-
млађе од 37 година и нискообразоване, старије од 37 година и ниже образоване, млађе од 37 година 
са високим степеном образовања и старије од 37 који су високообразовани.
Пре примене третмана ћемо за сваког испитаника проверити знање једним пробним тестом из математике, 
а затим ћемо у свакој од четири групе које се називају стратуми насумично одабрати људе који ће 
приступити курсу. Након завршетка курса ћемо извршити поновно тестирање и конструисати логистичку 
регресију која ће приказати зависност резултата финалног теста у односу на индикатор да ли је 
испитаник учествовао у курсу, као и резултат пре-теста испитаника и стратум којем он припада.
Овај модел ће нам показати какав је утицај слушања курса, а те резултате може приказати и за сваки 
појединачан стратум.
\end{primer}

\subsubsection{Блок рандомизација}

Блок рандомизација функционише по сличном принципу као и стратификована, с тим да се код блок 
рандомизације испитаници деле у мање групе чији чланови не морају бити повезани било каквом 
заједничком карактеристиком, већ је само битно да у сваком од блокова буде подједнак број 
испитаника распоредђених у контролну и у третман груоу. Овај тип рандомизације обезбеђује да групе 
имају једнаку величину, али њена лоша страна лежи у томе што је димензија сваког блока унапред 
позната па је могуће предвидети наредну доделу. 

\begin{primer}
Узећемо исти пример са онлајн курсом из математике, само што ћемо испитанике овог пута поделизи у 
блокове. Кључна разлика је у томе што ћемо овај пут испитанике делити у групе без вођења рачуна о 
њиховој старости или нивоу образовања. Једино што ће нам бити битно је да сви блокови буду 
идентичне величине која је погодна за податке које имамо. Како имамо 800 испитаника, а к
ардиналност третмана је 2 (постоје само две опције - \textit{"слушао курс"} и 
\textit{"није слушао курс"}) број 8 се намеће као оптималан број чланова у једном блоку. 
Поделићемо испитанике у групе од по осморо и у свакој од њих ћемо имати по тачно четворо који 
слушају курс. Након што поново одржимо курс до краја и одрадимо завршно тестирање 
анализираћемо добијене резултате и открити да ли је овај курс био делотворан.
\end{primer}

\subsubsection{Кластер рандомизација}

Кластер рандомизација је облик рандомизације у ком, уместо појединаца, целе групе добијају 
заједнички третман. Узорак се подели у групе (школе, села, фабрике...) и у свакој од тих група, 
које се називају кластери, или сви појединци добијају третман или не добија нико.
Овакав облик рандомизације је често практичнији, јер није увек могуће рандомизовати сваког 
појединца, међутим, оваквом рандомизацијом се губи на ефективности величине узорка и повећава се 
варијанса пошто у многим кластерима појединиц међусобно "личе". 

\begin{primer}
Покушавамо да сазнамо да ли би служење алкохолних пића на фудбалским стадионима у Србији повећало 
број гледалаца на утакмицама. Узећемо у обзир само фудбалске клубове из Суперлиге и Прве лиге Србије 
које заједно броје 32 клуба. Насумично ћемо, за сваки од клубова, одабрати да ли ће у асортиман 
хране и пића на свом стадиону увести и алкохолна пића. У наредном периоду бележимо посете на утакмицама 
које играју као домаћини и испитујемо да ли постоји значајна разлика у повећању посете код екипа 
које су увеле алкохолна пића у односу на оне које нису.
\end{primer}

\subsubsection{Адаптивна рандомизација}

Код адаптивне рандомизације се током извођења експеримента мења вероватноћа добијања третмана на 
основу до тада добијених резултата. На примеру са тестирањем два лека би се то спровело тако што 
онај лек који је до неког тренутка имао боље дејство дајемо наредним испитаницима у већем 
(углавном пропорционалном) броју случајева. На овај начин можемо минимизовати број испитаника са 
лошијим третманом што јесте веома позитивно, али је ипак примена адаптивне рандомизације веома 
компликована имајући у виду да су за њу потребни напредни статистички модели и детаљна анализа 
до тада спроведеног експеримента.

\subsubsection{Степенаста рандомизација}

Степенаста рандомизација се спроводи тако што на крају сви испитаници добију третман, али не сви 
у исто време. Овакву рандомизацију вршимо када је из етичких разлога непожељно да неко остане без 
третмана и она нам може помоћи да уочимо временску динамику идентификације третмана. 
Ипак, коришћење степенасте рандомизације захтева сложенију анализу због мењања резултата кроз 
време, а може довести и до појаве да испитаници који дуже чекају на третман промене понашање и 
реакцију на исти. 

\subsubsection{Повезана рандомизација}

Код повезане рандомизације се сви испитаници распоређују у парове који би требало да буду 
међусобно најсличнији (нпр по полу, годинама итд) и онда се у сваком пару насумично бира 1 од 2 
испитаника који ће добити третман, док онај други неће.
Ово је јако добро за контролу хомогеничности и смањивање варијансе, али је у великим и 
хетерогеним узорцима јако тешко наћи подобне парове, а у случају испадања (\textit{drop out}) 
једног члана у неком пару то у потпуности нарушава структуру парова. 

Јасмо је да конкретан тип рандомизираног експеримента, то јест, начин на који ћемо извршити 
рандомизацију зависи од конкретне ситуације, односно, података које поседујемо о испитаницима и 
броју испитаника и да не постоји универзалан одговор на питање која рандомизација је најбоља..

\subsection{Статистичка анализа рандомизираног експеримента}

Статистичка анализа рандомизираних експеримената покрива основне принципе процене ефекта третмана 
и аналитику процеса који се одвијају током једног оваквог истраживања.
//Један од најопштијих начина да утврдимо да ли је примењени третман остварио икакав ефекат је да 
неким од статистичких тестова тестирамо хипотезу да је АТЕ=0. За тестирање оваквих хипотеза се 
најчешће користи t-тест где је t-статистика једнака 
$t=\frac{ATE}{SE(ATE)}=\frac{Y_{tretman}-Y_{kontrola}}{\sqrt{s_{tretman}^2/n_{tretman}+s_{kontrola}^2/n_{kontrola}}}$, 
док бисмо у случају примене више третмана истовремено под кореном у имениоцу имали дисперзије и 
величине сваког од њих. У наредном примеру ћемо применом t-теста испитати да ли је примењени 
третман имао ефекта тестирањем хипотезе $H_0$:ATE=0 наспрам алтернативне $H_1:ATE\neq0$.

\begin{primer}
Вратимо се на испитивање висине код деце која пију козије/кравље млеко. Имамо податке о висинама и 
за контролну и за третман групу. Применом t-теста ћемо проверити да ли је хипотеза да је АТЕ+0 
тачна или је p-вредност мања од значајне, што би значило да прихватамо алтернативну хипотезу 
да је АТЕ различито од нуле и да самим тим козије млеко има утицаја на раст деце. Како смо за пример 
користили вештачке податке, у само једном од модела из тог примера (М3) је пијење козијег млека 
имало ефекта на висину деце.
\end{primer}

Поред t-теста, можемо користити и F-тест/АНОВА тест за које је потребно израчунати варијансу 
између група, али и унутрашњу варијансу у свакој од група. Већи број учесника позитивно утиче на 
моћ теста јер смањује стандардну грешку, а на повећање прецизности процене утиче и мања варијанса 
грешке па су из тог разлога корисни експерименти са блок или стратификованом рандомизацијом јер 
је код њих варијанса грешшке смањена. АНОВА тестови се користе код експеримената који имају више 
третманских група како не бисмо више пута примењивали t-тест, за сваки пар група по један, већ 
тим једним тестом истовремено оценили разлику свих група..

Након што АНОВА тест утврди да постоји значајна разлика између третманa потребно је одредити које 
се тачно групе разликују. Како бисмо контролисали укупну вероватноћу грешке првог реда спроводимо, 
такозване, \textit{post-hoc} тестове који ће нам показати међусобне разлике за сваке две групе, 
без да користимо велики број t-тестова. У наставку ћемо се упознати са неким од најпознатијих 
облика \textit{post-hoc} теста.
//Фишеров ЛСД (\textit{Least Significant Difference})је један од најједноставнијих облика пост-хок 
тестова и радимо га ако је АНОВА тест показао значајан резултат.
Он се заснива на серији т-тестова код којих при поређењу две групе сматрамо да између две групе 
постоји разлика ако је апсолутна вредност разлике њихових просека већа од променљиве \textbf{LSD} 
која се добија по формули:
$$LSD=t_{\frac{\alpha}{2},df_w}*\sqrt{MS_w\cdot(\frac{1}{n_1}+\frac{1}{n_2})}$$ 
где су \begin{itemize} 
    \item $t_{\frac{\alpha}{2},df_w}$ - критична вредност t-статистике са степеном слободе 
    $df_w$ истим као и код АНОВА-е, 
    \item $MS_w$ - процена варијансе грешке унутар групе, 
    \item $n_1$ и $n_2$ - димензије група које се пореде
\end{itemize}

Бонферони корекција је конзервативна метода која се користи код експеримената са малим бројем 
група. Она решава проблем повећања грешке првог реда тако што задржава укупан ниво значајности.
Проблем код спровођења великог броја t-тестова је акумулација њихових нивоа значајности, а ова 
корекција за свако појединачно поређење поставља нови ниво значајности 
$\alpha_{bonf}=\alpha/k$, где k ожначава број парова. На овај начин, укупан ниво значајности 
остаје исти (\alpha), али за велики број група ће нови нивои значајности сваког појединачног 
тестирања постати премали што може довести до прихватања хипотеза које би требале бити одбијене.

Тукидов ХСД (\textit Honestly Significiant Difference) тест је још један широко употребљаван вид 
пост-хок теста који се примењује на експерименте са већим бројем група.
Овај тест јединицу за поређење узима коришћењем \textit{Tukey Q} расподеле и она је у следећем 
облику:
$$HSD=q_{\alpha,k,df_w}\cdot\sqrt{\frac{MS_w}{2}\cdot(\frac{1}{n_1}+\frac{1}{n_2})}$$ 
у којој је са $q_{\alpha,k,df_w}$ дата критична вредност, а k представља број група.
Овај пост-хок тест се истиче по томе што нам приликом покретања у \textbf{R}з-у исписује табелу у 
којој за сваки пар група можемо видети и апсолутну разлику њихових средина, али и интервале 
поверења и p-вредности за те разлике.

Поред наведених постоје и многи други пост-хок тестови као што су: Шефеов, Данканоов... 

\begin{primer}
Узмимо поново пример у ком смо тестирали ефекат три различита ђубрива на принос парадајза. Применићемо 
АНОВА тест на податке о приносу и утврдити да ли постоји значајна разлика међу ђубривима. Уколико 
је p-вредност мања од 0.05 можемо рећи да разлика постоји и искористићемо и Фишеров ЛСД и Тукидов 
ХСД тест да откријемо које се тачно групе разликују. Код Фишеровог теста као испис можемо добити 
тачан списак ђубрива која се међусобно разликују по утицају, док ћемо код Тукидовог ХСД теста као 
крајњи резултат добити табелу са детаљним информацијама за све парове.
\end{primer}

Као што смо се до овог дела упознали, рандомизирани експерименти могу имати примену у великом 
броју области свакодневног живота, а пре него што пређемо на причу о опсервационим студијама, 
направићемо малу рекапитулацију свега што смо рекли о рандомизираним експериментима.
У питању су експерименти код којих се на насумичан начин одређује да ли ће неки испитаник примити 
третман, а у случају експерименаза са више третмана и који третман ће ко од испитаника примити.
Највећа предност оваквих истраживања је што тај насумичан одабир у потпуности елиминше утицај 
осталих варијабли на коначан исход и тако на непристрасан начин процењује ефекат третмана.
Код рандомизираних експеримената лако можемо донети закључке о томе да ли је дејство третмана 
значајног интензитета,
Такође, овакви експерименти су јако флексибилни па нам омогућавају да их спроводимо на много 
различитих начина, у зависности од конкретне ситуације, врсте третмана и података које поседујемо 
о испитаницима.
Ипак у неким случајевима из етичких или практичних разлога није могуће вршити рандомизацију.
Осим тога, рандомизирани експерименти често немају потребну спољну валидност. 
Ова два проблема се могу решити спровођењем опсервационих студија.

\newpage



\section{Опсервационе студије}


\end{document}

