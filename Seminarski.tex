\documentclass[12pt, a4paper]{article}

\usepackage{authblk}
\usepackage[utf8]{inputenc}
\usepackage[T2A]{fontenc}
\usepackage[serbianc]{babel}
\usepackage{hyperref}

\renewcommand\Authsep{\par}
\renewcommand\Authands{\par}

\title{Увод у каузално (узрочно) закључивање}
\author{Александра Новаковић 368/22}
\author{Милица Ињац 338/18}
\author{Бојан Корда 121/19}
\affil{Математички факултет, Универзитет у Београду}
\date{\today}

\begin{document}
\maketitle
\newpage

\tableofcontents
\newpage

\section{Фундаментални проблем}
    \subsection{Увод}
    \subsubsection{корелација не повлачи узрочност}
    \subsubsection{зашто је важно}
    \subsubsection{Главне разлике, истраживања, илустрација примером}
\subsection{Фундаментални проблем}
    \subsubsection{Рубинов каузални модел}
    Fundamentalni problem kauzalnog zaključivanja ogleda se u činjenici da za istu jedinku nikada ne možemo istovremeno posmatrati 
    oba potencijalna ishoda – onaj koji bi nastao ako je tretman primenjen i onaj koji bi nastao ako tretman nije primenjen. 
    Drugim rečima, kauzalni efekat na individualnom nivou ostaje uvek delimično nepoznat, jer je jedan od ishoda nužno kontrafaktualan.
     Ovaj okvir formalizuje Rubinov kauzalni model, u kojem se uvode potencijalni ishodi Y(1) i Y(0), a individualni kauzalni efekat 
     definiše kao njihova razlika. Međutim, pošto je samo jedan od ta dva ishoda posmatran, istraživači su usmereni na procenu 
     kontrafaktuala – onoga što bi se desilo u alternativnom scenariju.

Kako individualni efekti nisu direktno dostupni, u praksi se koriste agregatne mere kauzalnog efekta, poput prosečnog tretman efekta 
(Average Treatment Effect, ATE), definisanog kao E[Y(1)-Y(0)], ili prosečnog efekta na tretirane (Average Treatment Effect on the 
Treated, ATT), E[Y(1)-Y(0)|T=1]. Ove mere omogućavaju poređenje grupa i daju empirijski smisao istraživanjima u kojima se traži 
procena uzročnih uticaja. Da bi ovakva analiza bila validna, neophodna je pretpostavka SUTVA (Stable Unit Treatment Value Assumption),
 koja podrazumeva da ishod jedne jedinke ne zavisi od tretmana drugih jedinki i da ne postoje skrivene varijante tretmana.
Na taj način, fundamentalni problem kauzalnog zaključivanja objašnjava zašto uzročni efekti nisu neposredno posmatrani, 
već se moraju definisati kroz kontrafaktualne ishode i procenjivati pomoću statističkih metoda, oslanjajući se na pretpostavke
 poput SUTVA i mere efekta na nivou populacije.

 https://www.kurims.kyoto-u.ac.jp/~kyodo/kokyuroku/contents/pdf/1703-09.pdf

  whichwedrawarandomsampleof $n$
 units. Eachunit isabletobeexposedtoeitheratreatment
 oracontrol.Let $Z_{i}$
 representarandomvariableoftreatmentassignmentsothat $Z_{i}=1$ ifthe ith
 unit isassignedtothetreatmentgroupand $Z_{i}=0$ ifthe $i$
 thunit isassignedtothecontrolgroup.
 Thus, the ithunithastwopotentialoutcomes, $\}_{i} (1)$
 ifit isexposedtothetreatmentwhen $Z_{i}=1$,
 or $Y_{i}(0)$
 if it isexposedtothecontrolwhen $Z_{i}=0$ . Theobserveddataonthe ithunitconsistof
 thepair $(Z_{i},Y_{i})$ ,where
 $\}_{i} =Z_{i}1_{i}^{\nearrow}(1)+(1-Z_{i})Y_{i}(0)$
 .
 Theeffectcausedbytheteatmentforthe $i$
 thunit(relativetothecontrol),orsimplythetreatment
 effectforthe ithunit, isdefinedas thedifference $l_{i}^{f}(1)-Y_{i}(0)$ . Thisquantitymeasures thegain
 intheoutcomevariableundertheassignmenttothetreatmentrelativetothecontrol.Wesuppose
 thateachunitcanbeexposedtoonlythetreatmentorthecontrol, thereforewecanobserveeither
 $l_{i}(1)$
 or $Y_{i}(0)$ , butneverboth. That is,either $Y_{i}(1)$
 or $Y_{i}(0)$
 ismissingforthe ithunit, implying
 thatthetreatmenteffectforthe $i$
 thunit isnotobservable.ThisfactiscalledbyHolland $(|18|)$ the
 fundamentalptoblemofcausal ilference.

  Tostatisticallyovercomethefundamentalproblemofcausal inference, thefirstthingwedois
 toreplacethe inferentia]goalofestimatingthetreatmenteffectforan individualunitbyconsider
ingtheproblemofestimatingtheaveragetreatmenfeffect:
 $\theta=E\{Y_{i}(1)\}-E\{1_{i}^{\nearrow}(0)\}$ (1)

whercthcexpectationisassumedtobe independentof $i$
 .Wenotethatsincetheoperationalmean
ingsofthetworandomvariables $l_{i}(0)$
 and $1_{i}^{r}(1)$
 involvetherandomvariable $Z_{i}$
 , theexpectations
 $E\{Y_{i}(1)\}$ and $E\{\}_{i}^{-}(0)\}$
 thereforealmost $a$
 ]waysdependonthedistributionof $Z_{i}$
 , thatis.themech
anismofthetreatmentassignment.More $\exp$]icitly,wecanwrite
 $E\{Y_{i}(1)\}=E[E\{Y,\cdot(1)\}|Z_{i}]$
 andsimilar]yfor $E\{1_{i}^{r}(0)\}$
 .Theaveragetreatmenteffect $\theta$
 hasthepotential tobeestimatedbecause
 potentialoutcomes $1_{?}^{\nearrow}\cdot(1)$
 and $Y_{i}(0)$
 ondiffercntunitsmaynowbeusedtoestimatetheexpectations
 $E\{1_{i}^{r}(1)\}$
 and $E\{]_{i}^{r}(0)\}$
 . Toachievethisgoal, further $f^{\backslash }undamenta1$
 assumptionsonthetreatment
 assignmentmechanismarehoweverrequiredsincetheobservcddata $(Z_{i},l_{i}^{\nearrow})$
 onlyprovide infor
mationontheexpectations
 $E\{Y_{i}|Z_{i}=1\}=E\{Y_{i}(1)|Z_{i}=1\}$ and
 $E\{Y_{\dot{1}}|Z_{i}=0\}=E\{l_{7}^{r}(0)|Z_{i}=0\}$ .
 The fundamentalproblemofcausal inferencecanbeovercomebyconsideringtwosuchassump
tions,namelytheindependenceassumption([18])andtheassumptionofstrongignorabili $O^{}$
 ([29]).
 Bothconditions arenatural in the sense that theycanbederivedwhenone $considers^{\backslash}$
 the rela
tionsbetweentheexpectations $E\{Y_{i} (1)\},$ $E\{Y_{i}(0)\}$ andtheconditionalexpectations $E\{Y_{1}(1)|Z_{i}=$
 $1\},$ $E\{Y_{\dot{7}}(0)|Z_{i}. \infty0\}$
 . The independenceassumptionconcerns theclassical caseofrandomized
 experiment,whereweassume that the treatmentassignment $Z_{i}$
 is independentofthepotential
 outcomes $(l_{i}^{r}(1), l_{i}(0))$
 andal] otherpotential confoundingvariables. Cansal inference for ran
domizedexperiment is straightforwardbecauseunderthis independenceassumptionwehavethe
 basic identities
 $E\{l_{i} (1)\}=E\{Y_{i}(1)|Z=1\}$
 $E\{Y_{i}(0)\}=E\{\}_{i} (0)|Z=0\}$ .
 Thusthe independenceassumptionensurcs that
 $\theta=E\{Y_{i}(1)|Z=1\}-E\{Y_{i}(0)|Z=0\}$ (2)
 Sothesampledifferenceinthetwogroups inthiscasewillgiveanunbiasedestimatefor $\theta$

 Praktični kontekst: zašto nam je ATE važan – jer se u društvenim, biomedicinskim i ekonomskim istraživanjima retko bavimo 
 individualnim efektima, već nas zanima prosečan uticaj tretmana na populaciji.

Ograničenje ATE-a: naglasiti da iako ATE rešava problem kontrafaktuala na individualnom nivou, i dalje zavisi od pretpostavki 
o dodeli tretmana i mogućih konfaundera (pa zato koristimo randomizaciju, matching, IPW, itd.).

 EstimatingtheAverageTreatmentEffect ovde ide sada metode za procenu ATE 

 Fundamentalni problem kauzalnog zaključivanja → ne možemo posmatrati oba ishoda za istu jedinku (tretiran i netretiran).

Da bismo procenili kauzalni efekat (ATE) → moramo “rekonstruisati” kontrafaktualne ishode korišćenjem statističkih metoda.

Tu dolaze u igru metode poput regresije, matching-a, propensity score weighting-a – sve su to načini da što bolje aproksimiramo 
nedostajuće kontrafaktualne podatke.
 .


    \subsubsection{АТЕ (Average Treatment Effect), дефиниција}
    \subsubsection{проблем контрафактуала, немогућност да посматрамо оба света истовремено}
\subsection{Математичке основе иза концепта узрочности}
    \subsubsection{АТЕ (процена и математичке импликације)}
    \subsubsection{СУТВА}
    \subsubsection{алгоритми за процену ефекта}
\subsection{Утврђивање кауланости и експерименти (прелазак на следеће поглавље)}

\newpage



\section{Рандомизирани експеримент}
\newpage



\section{Опсервационе студије}


\end{document}

